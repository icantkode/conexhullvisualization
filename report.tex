\documentclass[10pt,twocolumn]{article}
\usepackage{times}
\usepackage{geometry}
\usepackage{graphicx}

\geometry{letterpaper, margin=1in}

\title{CS-2009 Pro Report}
\author{A. Siddique \and I. Shah \and M. Shakeel}
\date{\today}

\begin{document}
\maketitle

\section*{Abstract}
Briefly summarize the objectives, methods, and key findings of your project.

\section*{Introduction}
This project investigates the practical application of geometric algorithms,
exploring their complexity, performance characteristics, and the methodologies employed in their implementation.
The primary focus of this attempt is to address two distinct geometric problems. Firstly, we investigate the problem of
determining whether two line segments intersect, implementing the methods discussed in lectures and one proposed by <Author Name>[reference#].
Secondly, the project tackles the Convex Hull problem, employing various algorithmic approaches to find the convex hull of a set
of planar points. The algorithms under study include Brute Force, Jarvis March, Graham Scan, Quick Elimination, and one
additional approach learned from <AuthName>[reference]. Through the implementation and demonstration of these algorithms, we
aim to explain the diffiuculties of Convex Hull problem-solving, showcasing different strategies and their implications on computational efficiency.
To facilitate a comprehensive understanding, this project is built upon a user-friendly interface allowing for interactive point
placement. The user interface is meticulously designed to illustrate each step of the algorithms,
providing a visual aid to accompany the theoretical discussions. Moreover, the report places a significant emphasis on clarifying
the calculation of time and space complexities for each algorithm, offering insights into their efficiency and scalability.
In the following sections, we present the programming design, experimental setup, detailed results, and discussions.

\section*{Programming Design}

This section outlines the key aspects of the design for our project. Our implementation is designed to provide a user-friendly interface for 
interacting with various geometric algorithms. Java serves as the primary programming language for this project, 
chosen primarily due to the authors' proficiency in its usage.

\subsection*{User Interface}

The user interface is designed so that the users can place a point at their place of choosing when executing both line intersection algorithms
and convex hull algorithms. Once the user has placed all the points, the user can indicate to the system to run the algorithm/s.

\subsection*{Visualization}

To enhance understanding, the project provides visualization for each step of the algorithms. This includes graphical representations of input
data, selection and checking criteria of the points, and final results.

\subsection*{Time and Space Complexity Calculation}

Time and Space complexities are calculated by subtracting the system time when the algorithm was initiated, from the system time
after the algorithm finished its execution in milliseconds. Space complexity is handled 

\subsection*{Code Structure}

The codebase follows a modular structure, with separate modules for each algorithm and a central module for handling user input and visualization. This design promotes code reusability and maintainability.

\subsection*{Libraries and Dependencies}

We leverage [Any Libraries or Dependencies Used] to expedite the implementation process and ensure the correctness and efficiency of our algorithms.

\emph{Note: Customize this section based on the specifics of your project, including the actual programming language, libraries, and dependencies you are using.}


\section*{Experimental Setup}

\section*{Results and Discussion}
Present your findings, including screenshots and discussions on the implemented algorithms. Compare their performance, execution times, and any other relevant metrics.

\section*{Conclusion}
Summarize the key findings of your project and discuss any insights gained.

\section*{References}
Include citations to relevant sources, papers, or documentation that you consulted during your project.

\end{document}
